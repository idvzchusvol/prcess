\documentclass{ctexart}
\usepackage[fleqn]{amsmath}
\usepackage{tabu}
\usepackage{amssymb}
\usepackage{fancyhdr}
\usepackage{graphicx}
\usepackage{geometry}
\usepackage{amsthm}
\usepackage{listings}
\usepackage{xcolor}
\usepackage{amssymb}
\usepackage{wasysym}
\lstset{
  numbers=left, 
  numberstyle= \tiny, 
  keywordstyle= \color{ blue!70},
  commentstyle= \color{red!50!green!50!blue!50}, 
  %frame=shadowbox, % 阴影效果
  rulesepcolor= \color{ red!20!green!20!blue!20} ,
  escapeinside=``, % 英文分号中可写入中文
  xleftmargin=2em,xrightmargin=2em, aboveskip=1em,
  framexleftmargin=2em
} 
\setCJKmonofont{Consolas} 

\setlength{\textwidth}{14cm}
\setlength{\textheight}{20cm}
\setlength{\hoffset}{0cm}
\setlength{\voffset}{0cm}

\setlength{\parindent}{2em}                 
\setlength{\parskip}{3pt plus1pt minus1pt} 
\renewcommand{\baselinestretch}{1.2}        
\setlength{\abovedisplayskip}{2pt plus1pt minus1pt}     
\setlength{\belowdisplayskip}{6pt plus1pt minus1pt}     
\setlength{\arraycolsep}{2pt}  
\geometry{left=3.0cm,right=3.0cm,top=2.5cm,bottom=2.5cm} 

\allowdisplaybreaks[4] 

\pagestyle{fancy}
\lhead{Schedule}
\chead{}
\rhead{Author Idvz}

\title{Solution}
\author{Idvz}
\begin{document}
\date{}
\maketitle



%\newfontfamily\fzyingbikaishu{方正静蕾简体}
%\setCJKfamilyfont{方正静蕾简体}

\pagenumbering{Roman}

\section{总计划}
\begin{itemize}
\item 12月计划\\
  \sout{莫比乌斯反演,狄利克雷卷积,杜教筛,FFT,NTT},2-SAT,线性基,莫队
\item 1月计划\\
  线段树(线段树分治,时间轴分治),李超线段树,CDQ分治,虚树,LCT维护子树信息,可持久化平衡树,树分治,dsu on tree
\item 2月计划\\
  manacher,trie树,AC自动机,后缀数组,回文树,后缀自动机
\end{itemize}
  

\section{12月计划}

莫比乌斯反演(10天),FFT(15天)。

12.12 完成莫比乌斯反演,杜教筛,狄利克雷卷积(用时10天)。

12.21 完成FFT,NTT\sout{分治FFT除外}(用时10天)。

\iffalse
\begin{table}  
  \begin{center}  
    \begin{tabular*}{12cm}{llll}
      Date & Name & Source  & Algorithm\\
      \hline
      17.12.3 & Zap & BZOJ & 反演\\
      & YY的Gcd & BZOJ & 反演\\
      \hline
      17.12.4 & 数字 & BZOJ  & 反演\\
      \hline
      17.12.5 & Crash的数字表格 & BZOJ  & 反演\\
      & 完全平方数 & BZOJ  & 反演+二分\\
      & 仪仗队 & BZOJ  & 欧拉函数\\
      & Jzatab & BZOJ  & 反演\\
      & 约数和个数 & BZOJ  & 反演\\
      \hline
      17.12.6 & Number Challenge & Codeforces  & 反演\\
      \hline
      17.12.7 & 数表 & BZOJ & 反演+树状数组\\
      & 能量采集 & BZOJ  & 反演\\
    \end{tabular*}  
  \end{center}  
\end{table}  
\fi
\end{document}
