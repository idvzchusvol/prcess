\documentclass{ctexart}
\usepackage[fleqn]{amsmath}
\usepackage{tabu}
\usepackage{amssymb}
\usepackage{fancyhdr}
\usepackage{graphicx}
\usepackage{geometry}
\usepackage{amsthm}
\usepackage{listings}
\usepackage{xcolor}
\usepackage{amssymb}
\usepackage{wasysym}
\lstset{
  numbers=left, 
  numberstyle= \tiny, 
  keywordstyle= \color{ blue!70},
  commentstyle= \color{red!50!green!50!blue!50}, 
  %frame=shadowbox, % 阴影效果
  rulesepcolor= \color{ red!20!green!20!blue!20} ,
  escapeinside=``, % 英文分号中可写入中文
  xleftmargin=2em,xrightmargin=2em, aboveskip=1em,
  framexleftmargin=2em
} 
\setCJKmonofont{Consolas} 

\setlength{\textwidth}{14cm}
\setlength{\textheight}{20cm}
\setlength{\hoffset}{0cm}
\setlength{\voffset}{0cm}

\setlength{\parindent}{2em}                 
\setlength{\parskip}{3pt plus1pt minus1pt} 
\renewcommand{\baselinestretch}{1.2}        
\setlength{\abovedisplayskip}{2pt plus1pt minus1pt}     
\setlength{\belowdisplayskip}{6pt plus1pt minus1pt}     
\setlength{\arraycolsep}{2pt}  
\geometry{left=3.0cm,right=3.0cm,top=2.5cm,bottom=2.5cm} 

\allowdisplaybreaks[4] 

%\setCJKmainfont{FZJingLeiS-R-GB}

%\renewcommand{\CJKglue}{\hskip 0pt plus 0.08\baselineskip}

\pagestyle{fancy}
\lhead{Noip2017总结}
\chead{}
\rhead{Pl}

\title{Solution}
\author{Idvz}
\begin{document}
\maketitle



%\newfontfamily\fzyingbikaishu{方正静蕾简体}
%\setCJKfamilyfont{方正静蕾简体}

\pagenumbering{Roman}

\section{Noip2017 总结}
这次的Noip的成绩并不令人满意,对我自己而言是一个非常大的教训,这反映了自己平时在训练中的一些问题。

第一个原因就是心态不稳,像Day1T1这种题目,刚开考的时候并没有好的想法,所以给自己的后续考试带来的很大的心理负担,以至于在写第二题的时候,脑子里时不时地想起第一题,心态上不稳定,让自己根本没有发挥出平时的水平,碰到这种局面不知道从何下手,还有其他的因素就是考试经验的不足,在平时很少遇到过这样的情况,而且是第一题卡了自己非常久的时间。

最重要的原因就是自己的实力不强,Day1T3没有任何想法,没有关注到最短路的性质,也不知道如何去转移Dp,其次Day2T2的正解其实非常近,但没有想到怎么去优化,白白丢掉了20分,Day2T3同样没有想到正解,这些都反映了自己的实力并不强,这是最主要的原因。只有自己达到了那样的水平,才能做出题目来,所以,在后续的训练中主要的是要训练自己的思维能力,转化模型的能力,像今年的联赛题只要想出了模型,剩下的就是把算法往上面套。

在之后的训练中,主要是从几个方面来强化自己的水平,最重要的就是思维能力,在做题的时候需要想这个题目有什么特点,它的算法是怎样的,是怎么想到这一点的,总结出来就是套路,再把模型推广,就能解决这一类问题,在之前虽然有这样的意识,但做得太少了,套路见得并不多。同时,也还要和别人多交流,能见到一些自己没有见过的套路。实力的提升需要沉下心来,把这一类问题的所有套路全部都学会,这样这个知识点才算是真正的学会了,在学完之后,一定需要总结,总结能给自己一些更多的思考,非常有利于自己的提升。

第二个就是多积累考试经验,不管是什么样的考试,只要考试,就认真的考,考后思考自己的考试策略到底对不对,再及时调整,同时也还要克服自己在心态上面的问题,一定要让自己的心态稳下来。也还要思考自己考场上的做法有什么不足,或者为什么没有想到,再去找几道类似的题目做,总结出套路。

以这样的联赛分,给自己的省选带来了不利,但是最重要的是自己实力的提升。只剩下五个月的时间,还剩下许多算法需要学习,根据省选计划上面的内容,作出一些大致的计划。

\begin{itemize}
\item 数论,主要是莫比乌斯反演,FFT,NTT,生成函数,Polya定理,组合数学
\item 图论,网络流,树的点(边)分治
\item 数据结构,可持久化平衡树,莫队,CDQ分治,整体二分,以及各种数据结构的好题,套路题
\item 动态规划,四边形不等式优化,维护凸包,虚数上的Dp
\item 字符串,manacher,trie树,AC自动机,后缀数组,后缀自动机,回文树
\end{itemize}

特别地需要注意的是,在学习一个算法之后,要和pyh大佬交流,他会一些很少人见过的套路,其次学完之后一定要总结,还有就是时间真的不是很多了,需要提高自己的效率,把周日的时间也要利用上。平时可以记下几道题,在路上或者其他的时间让自己保持思考的状态。

沉下心来,下定决心,认真学。

\newpage

\end{document}
